\documentclass[12pt]{beamer}

% Packages:
\usepackage[utf8, utf8x]{inputenc}
\usepackage[american, french]{babel}
\usepackage{graphicx, aeguill, lmodern}
\usepackage[T1]{fontenc}
\usepackage[nice]{nicefrac}
\usepackage{ifpdf}
\usepackage{amsmath, amsfonts, amssymb}
\usepackage{longtable}
\usepackage{listings}
\usepackage{pgf,pgfarrows,pgfnodes,pgfautomata,pgfheaps,pgfshade, xcolor}

% Beamer tuning:
\usetheme{Copenhagen}
\usefonttheme[onlymath]{serif}
\beamertemplatenavigationsymbolsempty
\beamertemplatetransparentcovereddynamic

% LaTeX macros:
\newcommand{\ofgfr}[1]{\og{}#1\fg{}}
\newcommand{\ofgen}[1]{``\begin{otherlanguage}{american}#1\end{otherlanguage}''}
\newcommand{\so}[1]{\begin{flushright}\structure{$\rightarrow$ #1}\end{flushright}}

% Images:
\pgfdeclareimage[height=5mm]{amg-dev-logo}{images/amg-developpement-logo}
\pgfdeclareimage[height=5mm]{gpdis-ecommerce-logo}{images/gpdis-ecommerce-logo}
\pgfdeclareimage[width=15mm]{sudweb-logo}{images/sudweb-logo}
\pgfdeclareimage[width=60mm]{sudweb-big-logo}{images/sudweb-big-logo}
\pgfdeclareimage[width=70mm]{languages}{images/languages}
\pgfdeclareimage[width=40mm]{prozac}{images/prozac}

% Titlepage:	
\title{Vers des langues de programmation ?}
\subtitle{SudWeb 2011}
\author[Nicolas Dubois]{Nicolas Dubois - @duboisnicolas}
\date{Vendredi 27 mai 2011}
\institute{Lightning Talk}
\logo{\pgfuseimage{sudweb-logo}}

\begin{document}

% Aliases:
\def\tm{\texttrademark}
\def\r{\up{\textregistered}}
\def\sw{SudWeb'11}

\begin{frame}
\titlepage
\end{frame}

\begin{frame}{Préambule}
Cette présentation est ma nouvelle \emph{arme absolue} \pause pour endormir 
mes enfants en moins de \alert{5 minutes} le soir !
\end{frame}

\begin{frame}{Préambule}
Bon courage !
\end{frame}

\begin{frame}{Question initiale}{Paradoxe}
\begin{block}{Question initiale}
Pourquoi faut-il \alert{plusieurs années d'expérience} sur un \alert{langage} de programmation ?
\end{block}
\end{frame}

\begin{frame}[fragile]{Définition d'un langage}
\begin{block}{Définition}
Langage = Syntaxe + Sémantique
\end{block}
\pause
\begin{exampleblock}{Exemple de langage}
\begin{description}
\item[Syntaxe :] 
\lstset{language={Java}, basicstyle=\scriptsize}
    \begin{lstlisting}
public class A extends B {
    // Some code...
}
    \end{lstlisting}
    
\item[Sémantique :] définition d'une classe d'objet A qui étend la classe d'objet B.
\end{description}
\end{exampleblock}
\end{frame}

\begin{frame}{Définition d'un langage de programmation}
\begin{block}{Définition}
Langage = Syntaxe + Sémantique
\end{block}
\pause
\begin{block}{Définition}
Langage de programmation = Syntaxe + Sémantique + ?
\end{block}
\end{frame}

\begin{frame}{Contexte}
\begin{block}{Assertion : « Je maîtrise PHP »}
\begin{enumerate}[<+->]
\pause
\item Environnement de développement ;
\item Environnement de production ;
\item API ;
\item Écosystème ;
\item La communauté liée au langage.
\end{enumerate}
\end{block}
\end{frame}

\begin{frame}{Définition d'une langue}
\begin{block}{Définition}
Langage = Syntaxe + Sémantique
\end{block}
\begin{block}{Définition}
Langage de programmation = Syntaxe + Sémantique + ?
\end{block}
\pause
\begin{block}{Définition}
Langue = Langage + Histoire + Culture
\end{block}
\end{frame}

\begin{frame}{Langue de programmation}{Histoire}
Les inconsistences du langage ;
\pause
\begin{alertblock}{Exemple avec la librairie standard de PHP pour le traitement des 
chaînes de caractères}
\begin{enumerate}
\item str\_repeat()/str\_replace() vs strlen()/strpos()
\item strtolower()/strtoupper() vs bin2hex()/nl2br()
\item stripslashes() vs strip\_tags()
\item base64\_decode()/base64\_encode() vs urldecode()/urlencode()
\end{enumerate}
\end{alertblock}
\end{frame}

\begin{frame}[fragile]{Langue de programmation}{Culture}
\begin{block}{Culture}
Énormément d'éléments du langage sont culturels (car issus de conventions).
\end{block}
\pause
\begin{exampleblock}{Exemple de convention}
En python, il n'existe pas de mot clé pour instancier une classe.
\lstset{language={Python}, basicstyle=\scriptsize}
    \begin{lstlisting}
my_object = Object(params)
my_var = my_function(params)
    \end{lstlisting}
\end{exampleblock}
\end{frame}

\begin{frame}{Langue de programmation}{Langue vs Langage}
\begin{block}{Langue vs Langage}
\begin{description}[<+->]
\item[Langage de programmation :] la \alert{modalité de communication} entre l'homme et la machine ;
\item[Langue de programmation :] une instanciation de cette modalité au travers une histoire et une culture.
\end{description}
\end{block}
\pause
\begin{exampleblock}{Exemple de langue de programmation}
Java, PHP, Python, Ruby, \emph{etc}, sont des \alert{langues de programmation}.
\end{exampleblock}
\end{frame}

\begin{frame}{Conclusion}
\begin{block}{À quoi ça sert ?}
\ofgfr{Heu, ouais cool. Ça sert à quoi en fait ?\\\pause --- À rien !}
\end{block}
\end{frame}

\begin{frame}{Fausse fin}
\begin{center}
\huge{Merci}
\end{center}
\end{frame}

\begin{frame}{Conclusion}
\pause
\begin{alertblock}{Définition d'un développeur web}
Un développeur web n'est pas un \alert{bidouilleur de langages de programmation}.
\end{alertblock}
\pause
\begin{block}{Définition d'un développeur web}
Un développeur web est un \alert{polyglotte} qui met en œuvre plusieurs \alert{langues de programmation}.
\end{block}
\end{frame}

\begin{frame}{Vraie fin}
\begin{center}
\Huge{Merci}
\end{center}
\end{frame}

\end{document}
