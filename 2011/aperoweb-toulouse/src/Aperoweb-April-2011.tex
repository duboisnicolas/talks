\documentclass[notes]{beamer}

% Packages:
\usepackage[utf8, utf8x]{inputenc}
\usepackage[american, french]{babel}
\usepackage{graphicx, aeguill, lmodern}
\usepackage[T1]{fontenc}
\usepackage[nice]{nicefrac}
\usepackage{ifpdf}
\usepackage{amsmath, amsfonts, amssymb}
\usepackage{longtable}
\usepackage{listings}
\usepackage{pgf,pgfarrows,pgfnodes,pgfautomata,pgfheaps,pgfshade, xcolor}

% Beamer tuning:
\usetheme{Warsaw}
\useinnertheme{rectangles}
\usefonttheme[onlysmall]{structurebold}
%\setbeamercovered{transparent}
\beamertemplatenavigationsymbolsempty
\beamertemplatetransparentcovereddynamic

% LaTeX macros:
\newcommand{\ofgfr}[1]{\og{}#1\fg{}}
\newcommand{\ofgen}[1]{``\begin{otherlanguage}{american}#1\end{otherlanguage}''}
\newcommand{\so}[1]{\begin{flushright}\structure{$\rightarrow$ #1}\end{flushright}}

% Images:
\pgfdeclareimage[height=5mm]{amg-dev-logo}{images/amg-developpement-logo}
\pgfdeclareimage[height=5mm]{gpdis-ecommerce-logo}{images/gpdis-ecommerce-logo}
\pgfdeclareimage[width=15mm]{sudweb-logo}{images/sudweb-logo}
\pgfdeclareimage[width=60mm]{sudweb-big-logo}{images/sudweb-big-logo}
\pgfdeclareimage[width=70mm]{languages}{images/languages}
\pgfdeclareimage[width=40mm]{prozac}{images/prozac}

% Titlepage:
\title[Vers des langues de programmation ?]
    {Vers des langues de programmation ?\\
    \tiny{(version longue, mais pas trop)}}
\subtitle{SudWeb'11 - Lightning Talk}
\subject{Lightning Talk}
\author[Nicolas Dubois - @duboisnicolas]
    {Nicolas Dubois\\{\small <nicolas.c.dubois@gmail.com>}}

\date{Mardi 26 avril 2011 / Apéroweb Toulouse}
\logo{\pgfuseimage{sudweb-logo}}

\begin{document}

% Aliases:
\def\tm{\texttrademark}
\def\r{\up{\textregistered}}
\def\sw{SudWeb'11}

\begin{frame}
\titlepage
\end{frame}

\section{Introduction}

\subsection{Préambule}

\begin{frame}{Préambule}{Ça va être chaud !}
\pause
\begin{itemize}[<+->]
\item Désolé j'ai pas assez dormi (4h-7h) ;
\item En plus, j'ai pas pu aller chez le coiffeur samedi midi. J'ai une coupe de ouf !
\item J'sais même pas si j'ai fini mes slides, on va voir…
\item Ouais je craque un peu…
\item \alert{Désolé…}
\end{itemize}
\end{frame}

\begin{frame}{Préambule}{Au cas où…}
\pause
À priori, c'est une présentation pour les développeurs…\\
\pause
\vspace*{1cm}
Pour les autres, El Deseo propose :
\pause
\begin{itemize}[<+->]
\item les bieres corses à 3 EUR\footnote{J'ai voulu frimer et faire mes slides en \LaTeX{}, mais du coup, en standard, j'ai même pas
le signe Euro, trop la honte !} ;
\item les tapas entre 4.5 et 9 EUR.
\end{itemize}
\end{frame}

\subsection{Présentation}

\begin{frame}{Présentation}{Curriculum vit\ae}
    \pause
    \begin{itemize}[<+->]
    \item Nicolas Dubois
    \item Web Developer depuis 7 ans (eCommerce, agences web)
    \item Actuellement chez AMG Développement, Groupe GPdis
        \begin{itemize}
        \item Discounteo.com
        \item Pulsat.fr
        \item Villatech.fr
        \item VPCBoost.com
        \end{itemize}
    \end{itemize}
\end{frame}

\begin{frame}{Présentation}{Technologie}
    \pause
    \begin{itemize}[<+->]
    \item PHP
        \begin{itemize}
        \item Magento ;
        \item symfony 1.4 ;
        \item Zend Framework 1.8 (oui c'est un peu vieux).
        \end{itemize}
    \item Python/Django
    \end{itemize}
    \pause
    \begin{center}
    \pgfuseimage{prozac}
    \end{center}
\end{frame}

\begin{frame}{Présentation}{Traitement automatique des langues}
    \pause
    Plusieurs stages dans le traitement automatique des langues :
    \pause
    \begin{itemize}[<+->]
    \item LORIA (Nancy, Équipe Langue \&{} Dialogue) ;
    \item Valoria (Vannes, Équipe RIMH) ;
    \item IMAG/Liebniz (Grenoble, Équipe MAGMA).
    \end{itemize}
\end{frame}

\subsection{Contexte}

\begin{frame}{Contexte}{\sw, Nimes}
    Conférence retenue pour \sw, sous la forme d'un \alert{Lightning Talk} (5 minutes).
    \begin{center}
    \pgfuseimage{sudweb-big-logo}
    \end{center}
    \pause
    \alert{\bf $\rightarrow$ Il reste des places pour SudWeb : \url{http://sudweb.fr} !}
\end{frame}

\section{Rappels}

\subsection{Langage machine}

\begin{frame}{Langage machine}

    \begin{block}{Langage machine}
    Une machine ne comprend qu'un langage binaire : les instructions machines.
    \end{block}

    \pause

    \begin{exampleblock}{Exemple d'instruction machine}
    \verb!10110000 01100001!
    \pause
    Ce qui signifie :
    \pause
    \begin{description}[<+->]
    \item [\verb!10110000!] instruction pour charger en mémoire ;
    \item [\verb!01100001!] nombre 97 codé en binaire.
    \end{description}
    \end{exampleblock}


    \begin{alertblock}{Problèmes}
    \begin{itemize}[<+->]
    \item Nombre d'instructions énorme !
    \item Limite illisible !
    \item Quasi-impossible à debugger !
    \end{itemize}
    \end{alertblock}

\end{frame}

\subsection{L'assembleur}

\begin{frame}[fragile]{L'assembleur}{Première couche logicielle}
    \begin{block}{Définition}
    Représentation mnémonique des instructions machines qui permet d'écrire ces
    dernières plus facilement.
    \end{block}
    \pause
    \begin{exampleblock}{Traduction de 10110000 01100001}
    \pause
    \lstset{language={[x86masm]Assembler}, basicstyle=\scriptsize}
    \begin{lstlisting}
movb $0x61,%al
    \end{lstlisting}
    \end{exampleblock}
\end{frame}

\begin{frame}[fragile]{L'assembleur}{Première couche logicielle}
    \pause
    \begin{exampleblock}{Écrire \ofgfr{Bonjour} en assembleur}
    \lstset{language={[x86masm]Assembler}, basicstyle=\scriptsize}
    \begin{lstlisting}
        .globl _start
BONJ:   .ascii  "Bonjour\n"
 _start: mov     $4      , %eax
         mov     $1      , %ebx
         mov     $BONJ   , %ecx
         mov     $8      , %edx
         int     $0x80

         mov     $1      , %eax
         mov     $0      , %ebx
         int     $0x80
    \end{lstlisting}
    \end{exampleblock}
\end{frame}

\begin{frame}[fragile]{L'assembleur}{Première couche logicielle}
    \begin{alertblock}{Problèmes}
    \pause
    \begin{itemize}[<+->]
    \item Nombre d'instructions énormes (bis) ;
    \item Énormément lié à l'architecture du microprocesseur ;
    \item Effort cognitif trop important.
    \end{itemize}
    \end{alertblock}
\end{frame}

\subsection{Langage de programmation}

\begin{frame}{Langage de programmation}{Définition}
    \pause
    \begin{block}{Définition}
    Le langage de programmation est l'interface entre le langage machine et le
    langage naturel humain.
    \end{block}
\end{frame}

\begin{frame}{Langage de programmation}{Schéma (QUI DÉCHIRE TOUT)}
    \begin{center}
    \pgfuseimage{languages}
    \end{center}
\end{frame}

\section{Langage vs. Langue}

\subsection{Langage}

\begin{frame}{Langage}{Définition}
    \pause
    D'un point de vue linguistique, \pause
    \begin{center}
    Langage = Syntaxe + Sémantique
    \end{center}
\end{frame}

\subsection{Langue}

\begin{frame}{Langue}{Définition}
    \pause
    D'un point de vue linguistique,\pause
    \begin{center}
    Langue = Langage + Histoire + Culture
    \end{center}
\end{frame}

\subsection{Inspiration}

\begin{frame}{Inspiration}{Cas d'une application chez GPdis}
    \pause
    \begin{itemize}[<+->]
    \item Nouvelle application qui va agréger :
        \begin{itemize}[<+->]
        \item ERP ;
        \item CRM ;
        \item Plateforme Magento ;
        \item Prestataire logistique.
        \end{itemize}
    \end{itemize}
    \pause
    \begin{alertblock}{Formation}
    L'équipe de développement va avoir une formation de 3 jours sur :
    \pause
    \begin{itemize}[<+->]
    \item Flex (UI) ;
    \item Cobol (API).\verb!</troll>!
    \end{itemize}
    \end{alertblock}
\end{frame}

\subsection{Portée d'un langage de programmation}

\begin{frame}{Portée d'un langage de programmation}
    \pause
    \begin{alertblock}{En réalité}
    Ça n'est pas 3 jours de formation, \pause mais plutôt 3 ans d'expérience.
    \end{alertblock}
    \pause
    \begin{itemize}[<+->]
    \item Environnement de développement ;
    \item Environnement de production ;
    \item API du langage ;
    \item Écosystème du langage.
    \end{itemize}
\end{frame}

\section{Vers des langues de programmation}

\subsection{Évolution vers les langues naturelles}

\begin{frame}[fragile]{Les langages sont de plus en plus évolués}
    \pause
    Plus les années passent et plus les langages deviennent \alert{proche} du langage naturel.
\end{frame}

\begin{frame}[fragile]{Les langages sont de plus en plus évolués}
    \begin{exampleblock}{Exemple de code en PHP}
\lstset{ %
language=PHP,                % the language of the code
basicstyle=\scriptsize,       % the size of the fonts that are used for the code
}
\begin{lstlisting}
// Some code...
if($set->isUndefined())) {
    $set = new Set();
}
if($set->isEmpty()) {
    $set->addElement(10);
}
foreach($set->getElements() as $element) {
    print $element . PHP_EOL;
}
\end{lstlisting}
    \end{exampleblock}
\end{frame}

\begin{frame}[fragile]{Les langages sont de plus en plus évolués}
    \begin{exampleblock}{Même code sans bruit syntaxique}
    \pause
\begin{verbatim}
if set is undefined:
    set = new Set
if set is empty:
    set add element 10
foreach set get elements as element:
    print element
\end{verbatim}
    \end{exampleblock}
\end{frame}

\subsection{Histoire}

\begin{frame}{Histoire}{Notion de temps en philosophie}
\pause
L'informatique moderne a altéré la notion et la perception du temps en philosophie :
\pause
\begin{itemize}[<+->]
\item nano-technologies ;
\item vitesse de communication (emails, réseaux sociaux…).
\end{itemize}
\pause
Les langages de programmation ont une \alert{histoire} sur une échelle de temps différente de
celle des langues naturelles : \pause les langages de programmation évoluent plus vite que les langues.
\end{frame}

\begin{frame}[fragile]{Histoire}{Évolution d'un langage au cours de son histoire}
\begin{exampleblock}{Exemple de PHP 1.0}
\lstset{language=HTML, basicstyle=\tiny}
\begin{lstlisting}
<!--getenv HTTP_USER_AGENT-->
<!--ifsubstr $exec_result Mozilla-->
Hey, you are using Netscape!<p>
<!--endif-->

<!--sql database select * from table where user='$username'-->

<!--ifless $numentries 1-->
Sorry, that record does not exist<p>
<!--endif exit-->

Welcome <!--$user-->!<p>
You have <!--$index:0--> credits left in your account.<p>

<!--include /text/footer.html-->
\end{lstlisting}
\end{exampleblock}
\end{frame}

\subsection{Traduction inter-langages}

\begin{frame}{Traduction inter-langages}
\pause
Il est difficile de traduire d'un langage à un autre parce que :
\pause
\begin{itemize}[<+->]
\item il existe plusieurs paradigmes de programmation (procédural, orienté objet, événementiel, …) ;
\item il faut un référentiel commun pour traduire une même fonctionnalité.
\end{itemize}
\end{frame}

\begin{frame}{Traduction inter-langages}{Langage naturel}
\begin{exampleblock}{\ofgfr{Quand les poules auront des dents…}}
\pause
\begin{description}[<+->]
\item [en] \ofgfr{Quand les cochons voleront.} ;
\item [it] \ofgfr{Quand les ânes voleront.} ;
\item [ca] \ofgfr{Quand les vaches voleront.} ;
\item [ar] \ofgfr{Quand les vaches iront au pélerinage sur leurs cornes.}.
\end{description}
\end{exampleblock}
\end{frame}

\begin{frame}[fragile]{Traduction inter-langages}{Langage de programmation}
\begin{exampleblock}{Exemple de classe en PHP}
\lstset{ %
language=PHP,                % the language of the code
basicstyle=\scriptsize,       % the size of the fonts that are used for the code
}
\begin{lstlisting}
/**
 * Doc for my super class...
 */
class MyClass {
    protected $_element = array();
    public function getElement() {
        return $this->_element;
    }
    public function setElement($elt) {
        $this->_element = $elt;
    }
}
\end{lstlisting}
\end{exampleblock}
\end{frame}

\begin{frame}[fragile]{Traduction inter-langages}{Langage de programmation}
\pause
\begin{alertblock}{Mauvais exemple de classe en python}
\pause
\lstset{ %
language=python,
basicstyle=\scriptsize,
}
\begin{lstlisting}
class MyClass:
    """Doc for my super class..."""
    element = []
    def get_element(self):
        return self.element
    def set_element(self, elt):
        self.element = elt
\end{lstlisting}
\end{alertblock}
\end{frame}

\subsection{Culture de programmation}

\begin{frame}{Culture de programmation}{Convention d'écriture}
\pause
À chaque langage de programmation correspond une culture de programmation (ensemble d'usages/conventions).
\vspace*{1mm}
\pause
Comment nommer les méthodes : \pause
\begin{description}[<+->]
\item[Java] \verb!camelCase()! ;
\item[Python] \verb!notation\_underscore()! (à peu près) ;
\item[PHP] à priori \verb!camelCase()! mais on trouve de tout !
\end{description}
\end{frame}

\section{Conclusion}

\subsection{À quoi ça sert ?}

\begin{frame}{Conclusion}{À quoi ça sert ?}
\pause
\ofgfr{Ouais, ok, génial, mais ça sert à quoi ?}
\pause
\begin{itemize}[<+->]
\item à avoir une entrée gratuite pour \sw (pas de troll) ;
\item revaloriser le travail du développeur ?
\item quel est le langage ultime du futur ?
\end{itemize}
\end{frame}

\begin{frame}{Conclusion}{Questions}
\pause
\begin{center}
{\LARGE Questions ?\footnote{J'ai mis ça pour être poli, mais ça serait plus cool
si on buvait un coup…}}
\end{center}
\end{frame}

\end{document}
